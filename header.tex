%%---Main Packages-----------------------------------------------------------------------
\usepackage[english, ngerman]{babel}	%Mul­tilin­gual sup­port for LaTeX
\usepackage[T1]{fontenc}				%Stan­dard pack­age for se­lect­ing font en­cod­ings
\usepackage[utf8]{inputenc}				%Ac­cept dif­fer­ent in­put en­cod­ings
\usepackage{lmodern}                    %The newer Font-Set
\usepackage{textcomp}					%LaTeX sup­port for the Text Com­pan­ion fonts
\usepackage{graphicx} 					%En­hanced sup­port for graph­ics
\usepackage{float}						%Im­proved in­ter­face for float­ing ob­jects
\usepackage{ifdraft}                    %Let you check if the doc is in draft mode

%%---Useful Packages---------------------------------------------------------------------
\usepackage[pdftex,dvipsnames,table]{xcolor}  %Driver-in­de­pen­dent color ex­ten­sions for LaTeX
\usepackage{csquotes}                   %Simpler quoting with \enquote{}
\usepackage{siunitx} 					%A com­pre­hen­sive (SI) units pack­age
\usepackage{listings}					%Type­set source code list­ings us­ing LaTeX
\usepackage[bottom]{footmisc}			%A range of foot­note op­tions
\usepackage{footnote}					%Im­prove on LaTeX's foot­note han­dling
\usepackage{verbatim}					%Reim­ple­men­ta­tion of and ex­ten­sions to LaTeX ver­ba­tim
\usepackage[textsize=footnotesize]{todonotes} %Mark­ing things to do in a LaTeX doc­u­ment
\usepackage{booktabs}
\usepackage{lscape}
\usepackage{blindtext}
\usepackage{wrapfig}

%%---Tikz Packages-----------------------------------------------------------------------
\usepackage{standalone}
\usepackage{tikz}
\usepackage{circuitikz}
\usetikzlibrary{arrows}
\usetikzlibrary{calc}
\usetikzlibrary{intersections}

%%---Math Packages-----------------------------------------------------------------------
\usepackage{amsmath}					%AMS math­e­mat­i­cal fa­cil­i­ties for LaTeX
%\usepackage{amssymb}					%Type­set­ting symbols (AMS style)
%\usepackage{array}						%Ex­tend­ing the ar­ray and tab­u­lar en­vi­ron­ments
%\usepackage{amsthm}					%Type­set­ting the­o­rems (AMS style)

%%---Table Packages----------------------------------------------------------------------
\usepackage{tabularx}					%Tab­u­lars with ad­justable-width columns
%\usepackage{longtable}
\usepackage{multirow}					%Create tab­u­lar cells span­ning mul­ti­ple rows
\usepackage{multicol}					%In­ter­mix sin­gle and mul­ti­ple columns

%%---PDF / Figure Packages---------------------------------------------------------------
\usepackage{pdfpages}					%In­clude PDF doc­u­ments in LaTeX
\usepackage{pdflscape}					%Make land­scape pages dis­play as land­scape
\usepackage{subfig}					    %Fig­ures di­vided into sub­fig­ures

%%---Other Packages----------------------------------------------------------------------
%\usepackage{xargs}                     %De­fine com­mands with many op­tional ar­gu­ments

%%---Bibliography------------------------------------------------------------------------
\bibliographystyle{unsrt}

%%---Main Settings-----------------------------------------------------------------------
\graphicspath{{./graphics/}}			%Defines the graphicspath
%\geometry{twoside=false}				    %twoside=false disables the "bookstyle"
\setlength{\marginparwidth}{2cm}
\overfullrule=5em						%Creates a black rule if text goes over the margins => debugging


%%---User Definitions--------------------------------------------------------------------
%%Tabel-Definitions: (requires \usepackage{tabularx})
\newcolumntype{L}[1]{>{\raggedright\arraybackslash}p{#1}}    %column-width and alignment
\newcolumntype{C}[1]{>{\centering\arraybackslash}p{#1}}
\newcolumntype{R}[1]{>{\raggedleft\arraybackslash}p{#1}}

%%---Optional Package Settings-----------------------------------------------------------
%Listings-Settings: (requires \usepackage{listings}) => Example with Matlab Code
\lstset{language=Matlab,%
    basicstyle=\footnotesize\ttfamily,
    breaklines=false,%
    morekeywords={switch, case, otherwise},
    keywordstyle=\color{Blue},%
    tabsize=2,
    %morekeywords=[2]{1}, keywordstyle=[2]{\color{black}},
    identifierstyle=\color{Black},%
    stringstyle=\color{Purple},
    commentstyle=\color{Green},%
    showstringspaces=false,%without this there will be a symbol in the places where there is a space
    numbers=left,%
    numberstyle={\tiny \color{black}},% size of the numbers
    numbersep=9pt, % this defines how far the numbers are from the text
    %emph=[1]{word1, word2,...},emphstyle=[1]\color{red}
}							