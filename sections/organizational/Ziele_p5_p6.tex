\begin{landscape}
\section{Ziele P5/P6}
Die Ziele sind strikt aufgeteilt in die zwei Projekte 5 und 6. Darin enthalten sind die jeweiligen zu erreichenden Muss- und Wunschziele mit ihren quantifizierten Spezifikationen. Diese sind wichtig, da Ortsabhängig unterschiedliche Normwerte gelten und sich dieses Projekt grundsätzlich auf die Schweiz fokussiert.\\
\begin{table}[htbp]
  \centering
  %\small
  \caption{Ziele P5}
    \begin{tabular}{r|l|r|l|l}
          & \textbf{Ziel} & \multicolumn{1}{l|}{\textbf{Messbereiche}} & \textbf{Genauigkeiten} & \textbf{Einheiten} \\
    \toprule
    \multicolumn{1}{l}{\textbf{Mussziele P5}} & \multicolumn{1}{r}{} & \multicolumn{1}{r}{} & \multicolumn{1}{r}{} &  \\
    \toprule
    \multicolumn{1}{l|}{Sensoren} & Lufttemperaturmessung & \multicolumn{1}{l|}{[-20;60]} & $\pm$ 1 & $^\circ$C \\
\cline{2-5}          & Windgeschwindigkeitsmessung & \multicolumn{1}{l|}{[10;25]} & $\pm$ 1   & m/s \\
\cline{2-5} & Niederschlagsmenge &   Wasser    & $\pm$ 100 & ml/m$^2$ \\
    \hline
    \multicolumn{1}{l|}{Datenspeicherung} & Datenabfrage via PuTTY &   $\geq$ 9600    &       &  Bd/s\\
    \hline
    \multicolumn{1}{l|}{RTC} & Implementation &   Echtzeit    & $\pm$ 1   & s/Jahr \\
\bottomrule
\multicolumn{1}{l}{\textbf{Wunschziele P5}} & \multicolumn{1}{l}{} & \multicolumn{1}{l}{} & \multicolumn{1}{l}{} &  \\
    \toprule
    \multicolumn{1}{l|}{Sensoren} & Sonnenstunden Prototyp &   Echtzeit    &       & s \\
    \bottomrule
    \end{tabular}%
  \label{tab:ZieleP5}%
\end{table}%

Tabelle \ref{tab:ZieleP5} zeigt diverse Ziele im P5, unterteilt in Muss- und Wunschziele. Zu den Musszielen gehören die Lufttemperaturmessung, die Windgeschwindigkeitsmessung, die Niederschlagsmessung, die Implementation des RTC und die mögliche Datenabfrage via Putty vom Datenspeicher. Die Lufttemperatur soll zwischen -20 bis 60 $^\circ$C ermittelbar sein, mit einer Genauigkeit von $\pm$1 $^\circ$C. Die Windgeschwindigkeitsmessung soll vor allem stärkere Windgeschwindigkeiten erfassen, um vor Sturm warnen zu können, weshalb niedrigere Windgeschwindigkeiten vernachlässigt werden können. Die Windgeschwindigkeit soll zwischen 10 und 25 m/s auf $\pm$1 m/s genau gemessen werden. Die Niederschlagsmenge soll nur für Regenwasser bestimmt werden mit einer Genauigkeit von $\pm$100 ml/m$^2$. Als Wunschziel soll eine Möglichkeit getestet werden um Sonnenstunden zu detektieren, welche dann im P6 umgesetzt wird.\\

\newpage
% Table generated by Excel2LaTeX from sheet 'Tabelle1'
\begin{table}[htbp]
  \centering
  %\small
  \caption{Ziele P6}
    \begin{tabular}{l|l|r|r|r}
          & \textbf{Ziel} & \multicolumn{1}{l|}{\textbf{Messbereiche}} & \multicolumn{1}{l|}{\textbf{Genauigkeiten}} & \multicolumn{1}{l}{\textbf{Einheiten}} \\
    \toprule
    \multicolumn{1}{l}{\textbf{Mussziele P6}} & \multicolumn{1}{r}{} & \multicolumn{1}{r}{} & \multicolumn{1}{r}{} &  \\
    \toprule
    Speisung & Akkukapazität &       &       &  \\
\cline{2-5}          & Ladeschaltung Akku &       &       &  \\
\cline{2-5}           & Ladeschaltung Photovoltaik &       &       &  \\
    \hline
    Kommunikationsmodul & GPS   &       &       &  \\
\cline{2-5}          & Mobilfunk (SMS) &       &       &  \\
\hline
Sensoren & Sonnenstunden &       &       &  \\
    \bottomrule
    \multicolumn{1}{l}{\textbf{Wunschziele P6}} & \multicolumn{1}{r}{} & \multicolumn{1}{r}{} & \multicolumn{1}{r}{} &  \\
    \toprule
    Kommunikationsmodul & Mobilfunk (Website) &       &       &  \\
    \hline
    Speisung & Akku austauschbar &       &       &  \\
    \bottomrule
    \end{tabular}%
  \label{tab:ZieleP6}%
\end{table}%

\noindent
Tabelle \ref{tab:ZieleP6} zeigt diverse Ziele im P6, unterteilt in Muss- und Wunschziele. Diese Tabelle ist unvollständig und wird im P6 nachgeführt. Generell kann gesagt werden, dass die Speisung, das Kommunikationsmodul mit GPS und Mobilfunk, sowie die Sonnenstunden-Sensorik implementiert werden sollen. Als Wunschziele sind ein austauschbarer Akku und eine Website zur Datensicherung und ggf. grafischen Darstellung aufgeführt.
\end{landscape}
