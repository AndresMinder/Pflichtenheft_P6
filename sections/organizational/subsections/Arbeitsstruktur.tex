\subsection{Arbeitsstruktur}
Dieses Projekt umfasst viele unterschiedliche Komponenten von Firm- und Hardware. Selbst die Hardware befasst sich mit der analogen (z.B. Energieversorgung), wie auch mit der digitalen Domäne (z.B. Datenverwaltung). Auch die Firmware ist sehr Hardware bezogen, weshalb es sich nicht lohnt die Arbeitsverteilung klar zu strukturieren. Das Ziel ist also hauptsächlich, dass beide Parteien des Projektteams mit den unterschiedlichen Domänen konfrontiert werden. Daraus resultiert eine recht sporadische Arbeitsverteilung. Um aber Ansprechpartner für die Aufgabengebiete zu gewährleisten, werden die Verantwortlichkeiten von Teilgebieten auf beide Parteien verteilt.\\

\begin{table}[h]
\centering
\caption{}
\label{tab:verantwortlichkeitsgebiete}
\renewcommand{\arraystretch}{1.4}
	\begin{tabular}{l|l}
	\textbf{Wer} & \textbf{Verantwortlichkeitsgebiete} \\ 
	\hline 
	\multirow{4}{*}{Mischa Knupfer} & Pflichtenheft \\ 	 
	 & Energieversorgung \\ 	 
	 & Gehäuse \\ 	 
	 & Sensoren \\ 
	\hline 
	\multirow{4}{*}{Andres Minder} & Fachbericht \\ 	 
	 & Kommunikationsmodul \\ 	 
	 & Printherstellung \\ 	 
	 & Firmware \\ 
	\end{tabular} 
\end{table}

Nach Tabelle \ref{tab:verantwortlichkeitsgebiete} teilen sich die Gebiete grob in das \textit{Äussere} und \textit{Innere} der Wetterstation auf\footnote{rsp. in das von Aussen nach Innen wirkende und umgekehrt}.\\