\subsection{Kommunikation}
Die Kommunikation erfolgt grundsätzlich per E-Mail, ausser für Notfälle. Dafür sind die Telefonnummern aller Projektinstanzen noch zusätzlich in diesem Dokument hinterlegt (siehe Tabelle \ref{tab:kontaktinformationen}).\\

\begin{table}[htbp]
  \centering
  \small
  \renewcommand{\arraystretch}{1.4}
  \caption{Kontaktinformationen}
  \label{tab:kontaktinformationen}
    \begin{tabular}{l|l|l|l}
    \textbf{Projektinstanz} & \textbf{Name} & \textbf{E-Mail} & \textbf{Telefon} \\
    \toprule
    Auftraggeber/ & \multirow{2}[2]{*}{Prof. Dr. Taoufik Nouri} & \multirow{2}[2]{*}{\textcolor[rgb]{ .02,  .388,  .757}{taoufik.nouri@fhnw.ch}} & \multirow{2}[2]{*}{+41 79 218 38 55} \\
    Projektbetreuer &       &       &  \\
    \hline
    Experte & Patrick Strittmatter & \textcolor[rgb]{ .02,  .388,  .757}{patrick.strittmatter@actemium.ch} & +41 79 879 65 20 \\
    \hline
    Projektteam & Mischa Knupfer & \textcolor[rgb]{ .02,  .388,  .757}{mischa.knupfer@students.fhnw.ch} & +41 78 761 83 73 \\
    \hline
    Projektteam & Andres Minder & \textcolor[rgb]{ .02,  .388,  .757}{andres.minder@students.fhnw.ch} & +41 79 810 82 13 \\
    \end{tabular}%
  \label{tab:addlabel}%
\end{table}%

\vspace{0.5cm}

Im Verlaufe dieses Projektes wird alle zwei Wochen eine Sitzung mit Herrn Prof. Dr. Taoufik Nouri und dem Projektteam abgehalten. Darin werden aktuelle Angelegenheiten diskutiert und jegliche pendente Themen angesprochen. Für aufgetretene Probleme wird konstruktiv nach Lösungen für das weitere Vorgehen gesucht. \\

Die Sitzungseinladungen sind vom Projektteam aus zu verschicken, sowie auch die Sitzungen zu protokollieren. Jedes Protokoll wird innerhalb einer Woche nach der Sitzung per E-Mail vom Projektteam aus an alle Instanzen des Projektes gemäß Tabelle \ref{tab:kontaktinformationen} mit einer Aktionsliste\footnote{eine Liste mit Angaben, wer was in welchem Zeitraum zu erledigen hat} verschickt. Im darauffolgenden Protokoll wird die Annahme aller nötigen Instanzen dokumentiert.\\

Zwischen dem Projektteam und dem Experten wird keine weitere Kommunikation außer das Mitteilen der relevanten Dokumenten\footnote{Pflichtenheft, Sitzungsprotokolle und Fachbericht}, zwei Sitzungen (April \& Juli) und der Verteidigung am Schluss der Bachelor-Thesis erfolgen.