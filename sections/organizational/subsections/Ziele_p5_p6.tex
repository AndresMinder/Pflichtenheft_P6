\begin{landscape}
\subsection{Ziele}
Die Tabelle \ref{tab:ZieleP6} zeigt die diskreten Ziele des Projekts 6, rsp. jene der Bachelor-Thesis. Darin enthalten sind die jeweiligen zu erreichenden Muss-, Nicht- und Wunschziele mit ihren quantifizierten Spezifikationen.\\

\begin{table}[htbp]
  \centering
  \renewcommand{\arraystretch}{1.4}
  \caption{Ziele}
    \begin{tabular}{l|l|l|r|r}
          & \textbf{Ziel} & \multicolumn{1}{l|}{\textbf{Spezifikation}} & \multicolumn{1}{l|}{\textbf{Genauigkeit}} & \multicolumn{1}{l}{\textbf{Einheit}} \\
    \toprule
    \multicolumn{1}{l}{\textbf{Mussziele}} & \multicolumn{4}{r}{} \\
    \toprule
    \multirow{4}{*}{Speisung} & Akkukapazität (Li-Ionen Akku) & \multicolumn{1}{r|}{4'000\,$\leq$\,6'000} &  & mAh \\
\cline{2-5}           & Akkulaufzeit & \multicolumn{1}{r|}{$\geq$\,100} &   & h \\
\cline{2-5}          & Ladeschaltung Akku &       &       &  \\
\cline{2-5}           & Ladeschaltung Photovoltaik &       &       &  \\
    \hline
    \multirow{2}{*}{Kommunikationsmodule} & GPS-Modul   &  \multicolumn{1}{r|}{<\,5}  &       & Hz \\
\cline{2-5}          & GSM-Modul  &       &       &  \\
\hline
Sensoren & Sonnenstunden (Lichtstromdichte) & \multicolumn{1}{r|}{0.1 - 90'000} & & lx \\
    \bottomrule
    \multicolumn{1}{l}{\textbf{Nichtziele}} & \multicolumn{4}{r}{} \\
    \toprule
    & & & & \\
    \bottomrule
    \multicolumn{1}{l}{\textbf{Wunschziele}} & \multicolumn{4}{r}{} \\
    \toprule
    Kommunikationsmodule & Einbindung in IoT &       Bluetooth || WLAN &       &  \\
    \hline
    \multirow{3}{*}{Speisung} & Akku leicht austauschbar &       &       &  \\
\cline{2-5}  & Mittels USB ladbar & USB 2.0 (Mini-B || Micro-B) &       &  \\
\cline{2-5}  & Mittels DC-Ladekabel ladbar &      5.5 / 2.1mm DC-Stecker &       &  \\
    \bottomrule
    \end{tabular}%
  \label{tab:ZieleP6}%
\end{table}%
\todo[inline]{Es müssen noch Einträge in die Tabelle genommen werden. Sowie die Spezifikationen und weiteren Teile inkludieren. Ladeschaltung Akku Unterladungs- \& Überladungsschutz spezifizieren. Ladeschaltung Photovoltaik Ladestrom definieren. }
\end{landscape}
