\begin{landscape}
\subsection{Ziele}
Die Tabelle \ref{tab:ZieleP6} zeigt die diskreten Ziele dieses Projektes. Darin enthalten sind die jeweiligen zu erreichenden Muss-, Nicht- und Wunschziele mit ihren quantifizierten Spezifikationen.\\
\begin{table}[htbp]
  \centering
  \renewcommand{\arraystretch}{1.1} %Angepasst da sonst neue Seite
  \caption{Ziele}
    \begin{tabular}{l|l|l|r|r}
          & \textbf{Ziel} & \multicolumn{1}{l|}{\textbf{Spezifikation}} & \multicolumn{1}{l|}{\textbf{Genauigkeit}} & \multicolumn{1}{l}{\textbf{Einheit}} \\
    \toprule
    \multicolumn{1}{l}{\textbf{Mussziele}} & \multicolumn{4}{r}{} \\
    \toprule
  \multirow{3}{*}{Speisung} & Akkulaufzeit & \multicolumn{1}{r|}{$\geq$\,100} &   & h \\
    \cline{2-5}  & Mittels DC-Ladekabel ladbar &      5.5 / 2.1mm DC-Stecker &       &  \\
	\cline{2-5}           & Photovoltaik &    \multicolumn{1}{r|}{1}   &       & Akkuladungen/Tag \\
    \hline
  \multirow{2}{*}{Kommunikationsmodule} & GPS-Modul Standortupdate   &  \multicolumn{1}{r|}{<\,5}  &       & Minuten \\
	\cline{2-5}          & GSM-Modul  & SMS: Senden und Empfangen &       &  \\
\hline
Sensoren & Sonnenstunden (Lichtstromdichte) & \multicolumn{1}{r|}{0.1 - 90'000} & & lx \\
    \bottomrule
    \multicolumn{1}{l}{\textbf{Nichtziele}} & \multicolumn{4}{r}{} \\
    \toprule
     Sensoren aus Projekt 5& Keine Neukonzeption &  &  &  \\
     \hline
     Wartungsfreiheit & Wartungsfenster & \multicolumn{1}{r|}{>7} &  & Tage \\
    \bottomrule
    \multicolumn{1}{l}{\textbf{Wunschziele}} & \multicolumn{4}{r}{} \\
    \toprule
    Kommunikationsmodule & Einbindung in IoT &       Bluetooth || WLAN &       &  \\
    \hline
    \multirow{2}{*}{Speisung} & Akku leicht austauschbar &       &       &  \\
\cline{2-5}  & Mittels USB ladbar & USB 2.0 (Mini-B || Micro-B) &       &  \\

    \bottomrule
    \end{tabular}%
  \label{tab:ZieleP6}%
\end{table}%

Tabelle \ref{tab:ZieleP6} zeigt die Muss-, Nicht- und Wunschziele des Projekts auf. Die Akkulaufzeit und das Laden mittels DC-Ladekabel sind spezifiziert. Ausserdem wird ein volles Aufladen des Akkus innerhalb eines Tages via Photovoltaik als Ziel gesetzt. Dabei wird von der maximal möglichen Leistung der Photovoltaik ausgegangen, bei einer optimalen Sonneneinstrahlung von 10 Stunden. Das Update des Standorts wird spätestens alle 5 Minuten erfolgen und das Empfangen und Senden von SMS wird ebenfalls möglich sein. Nicht Ziel des Projekts ist es, die in einem vorhergehenden Projekt entworfene Sensoren neu zu konzipieren, sowie die mobile Wetterstation komplett Wartungsfrei zu machen. Eine wöchentliche Wartung ($\leq$ 7 Tage) wird als vertretbar bis notwendig erachtet. Als Wunschziel folgt die Einbindung der mobilen Wetterstation in das \textit{IoT}, ein leicht austauschbarer Akku, sowie die Möglichkeit, den Akku in der mobilen Wetterstation via USB-Kabel zu laden.

\end{landscape}
