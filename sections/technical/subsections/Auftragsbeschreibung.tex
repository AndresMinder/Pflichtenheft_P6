\section*{Auftragsbeschreibung}
Das Wetter spielt eine wichtige Rolle in der Agronomie. Regnet es nicht genug, müssen Pflanzen bewässert werden. Trifft auf ein Ort nur wenig Sonnenlicht, so sollten dort nicht die Pflanzen, welche viel Sonnenlicht brauchen, angebaut werden. Windet es zu stark, können Pflanzen beschädigt oder gar zerstört werden. Ist es Tagsüber heiss, so benötigen die Pflanzen mehr Wasser. Hiesige Bauern besitzen den Luxus von guten Wettervorhersagen dank dem Bundesamt für Meteorologie und Klimatologie (MeteoSchweiz). Dieser Luxus ist in anderen Ländern noch nicht gegeben. Prof. Dr. Nouri Taoufik ist aufgefallen, dass in tropischen Gegenden wie Südamerika oder teile Afrikas dieser Luxus ebenso fehlt. \\[0.5cm]
Aus diesem Grund soll eine kostengünstige, erweiterbare und mobile Wetterstation gebaut werden, welche diese Bauern unterstützt. Diese Wetterstation soll die Regenmenge, die Windstärke, die Lufttemperatur und die Sonnenstunden messen können. Ausserdem soll die Wetterstation mittels Photovoltaik unterstützt werden, und erhobene Daten via SMS abrufbar sein. \\[0.5cm]
Im Nachfolgenden Dokument werden unter anderem die Ziele dieses Projekts definiert, sowie das Gesamtkonzept näher erläutert. \todo{Die Auftragsbeschreibung muss überarbeitet werden.}
