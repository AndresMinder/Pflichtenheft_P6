\cleardoublepage
\thispagestyle{empty}
\section*{Auftragsbeschreibung}
Das Wetter spielt eine wichtige Rolle in der Agronomie. Aufgrund von Wetter- und Klimadaten können optimale Standorte für Pflanzen eruiert und Massnahmen zu deren Schutz getroffen werden. Hiesige Bauern besitzen den Luxus von guten Wettervorhersagen und Klimadaten dank dem Bundesamt für Meteorologie und Klimatologie (MeteoSchweiz). Dieser Luxus ist in anderen Ländern noch nicht gegeben. Prof. Dr. Taoufik Nouri ist aufgefallen, dass in tropischen Gegenden wie Südamerika oder teile Afrikas dieser Luxus ebenso fehlt. \\

Aus diesem Grund soll eine kostengünstige, erweiterbare und mobile Wetterstation gebaut werden, welche diese Bauern unterstützt. Diese Wetterstation muss die Regenmenge, die Windstärke, die Lufttemperatur und die Sonnenstunden messen können. Ausserdem sollte die Wetterstation mittels Photovoltaik unterstützt werden, sowie die erhobenen Daten via SMS abrufbar sein. \\

In einem ersten Projekt wurde die Sensorik zur Messung der Lufttemperatur, Luftfeuchtigkeit, Luftdruck, Regenmenge, Windstärke und Windrichtung implementiert, jedoch nicht vollständig verifiziert. Ausserdem wurde eine RTC implementiert, welche Zeitstempel liefert, um die erhobenen Daten zu datieren. Auf einer $\mu$SD-Karte können Daten gespeichert und über eine serielle Schnittstelle ausgegeben werden. Das ganze wird über eine MCU gesteuert. \\
\todo[inline]{Nachfragen ob so gut oder detailierter...}

Im Nachfolgenden Dokument werden unter anderem die Ziele dieses Projekts definiert, sowie das Gesamtkonzept näher erläutert. 

%\todo[inline]{Auftragsbeschreibung kontrollieren und Verbesserungsvorschläge machen.}