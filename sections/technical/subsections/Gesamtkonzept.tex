\subsection{Gesamtkonzept}
\label{chap:grundkonzept}
\begin{figure}[h]
	\centering
	\includegraphics[width=0.9\textwidth]{graphics/Konzeptdiagramme/Grundkonzept.PNG}	
	\caption{Grundkonzept}
	\label{fig:grundkonzept}
\end{figure}

\paragraph{Übersicht:}
Als Zentralrecheneinheit wird eine \textit{Micro-Controller-Unit (MCU)} verwendet. Dieser ist dafür verantwortlich, dass die Daten richtig verarbeitet und an das dementsprechende Modul weitergeleitet werden. Die Messdaten werden in digitaler Form vom Modul \textit{Sensoren} an die \textit{MCU} übertragen. Dieser fügt mit dem \textit{Real-Time-Clock (RTC)} einen Timestamp hinzu, wobei anschließend die Daten in der \textit{Datenspeicherung} nichtflüchtig gespeichert werden. Über das \textit{Kommunikationsmodul} können dann die Daten von Nutznießern abgefragt werden.\\

Das Gesamtkonzept ist, wie in der Abbildung \ref{fig:grundkonzept} grafisch dargestellt, modular aufgebaut. Auf die in diesem Projekt relevanten einzelnen Module wird folgend spezifischer eingegangen.\\

In einem früheren Projekt wurde bereits die Sensorik zur Ermittlung der Lufttemperatur, Luftfeuchtigkeit, Luftdruck, Windgeschwindigkeit, Windrichtung und Regenmenge online erworben und implementiert. Im Projekt 6 wird die Sensorik zur Ermittlung der Sonnenstunden implementiert, sowie die gesamte Sensorik verifiziert und gegebenenfalls optimiert. Nachfolgend wird die Art und Weise erläutert, wie die Messdaten ermittelt werden.\\

\subsubsection{Kommunikationsmodul}
\begin{figure}[h]
\centering
\includegraphics[scale=0.7]{graphics/Konzeptdiagramme/Kommunikationsmodul.PNG}
\caption{Kommunikationsmodul}
\label{fig:kommunikationsmodul}
\end{figure}
Abbildung \ref{fig:kommunikationsmodul} zeigt die verschiedenen Schnittstellen, über welche Daten mit der Umgebung (User) und \textit{MCU} ausgetauscht werden können. Im Rahmen eines vorhergehenden Projekts wurde das USB-Interface umgesetzt. Mobilfunknetz und GPS sind Teil von diesem Projekt.\\

\subparagraph{USB-Interface:}
Über dieses Interface kann mit dem System kommuniziert und interagiert werden. Ein serielles Terminal-Emulationsprogramm (wie z.B. PuTTY) wird dazu benötigt.\\

\subparagraph{Mobilfunknetz:}
Die mobile Wetterstation soll mittels SMS abgefragt werden können. Um dies zu bewerkstelligen, wird ein GSM-Modul benötigt, welches auf dem PCB integriert werden kann. Ein GSM-Modul benötigt eine SIM-Karte, welche zur Identifikation des Nutzers dient. Das Modul empfangt digitale Befehle via SMS und leitet diese über eine serielle Schnittstelle weiter zur eigenen MCU, worin diese Befehle verarbeitet werden und eine Antwort-SMS auslösen. Die Form der Antwort-SMS soll erst im weiteren Verlauf des Projekts festgelegt werden. Ferner wäre es von Vorteil die erhobenen Daten auf einen Server zu Uploaden, damit Daten für die Klimaforschung gesammelt werden können, was jedoch als Wunschziel definiert wurde. Es ist ebenso möglich die mobile Wetterstation in das \textit{IoT} (Internet of Things) zu implementieren, womit die Datenerfassung der Sensorik sich in Echtzeit verfolgen lässt, was jedoch ebenso ein Wunschziel des Projekts ist. \\

\subparagraph{GPS:}
Um den Standort der mobilen Wetterstation zu ermitteln, wird ein GPS-Modul implementiert. Ein GPS-Modul errechnet die Laufzeiten der Signale der Satelliten bis zum Empfänger. Dadurch, dass die Ausbreitungsgeschwindigkeit bekannt ist (Lichtgeschwindigkeit), kann auf die Entfernung zurückgerechnet werden. Mittels Triangulation (wenn die Entfernung zu 3 Satelliten bekannt ist) erfolgt schliesslich die Positionsbestimmung. Dazu muss das GPS-Modul über die Positionsdaten der Satelliten verfügen, die mit Hilfe bestimmter Daten errechnet werden, welche als Datensatz dem Empfängermodul vorliegen. Die Positionsbestimmung ist jedoch ebenso abhängig von Umgebungsfaktoren wie z.B. Berge oder hohe Gebäude, weshalb in diesem Projekt keine exakte Positionsbestimmung garantiert werden kann.\\
\subsubsection{Energieversorgung}
\begin{figure}[h]
\centering
\includegraphics[scale=0.5]{graphics/Konzeptdiagramme/Energieversorgung_1.PNG}
\caption{Energieversorgung des Systems mit Systemgrenze}
\label{fig:Energieversorgung_1}
\end{figure}
In Abbildung \ref{fig:Energieversorgung_1} ist die Systemgrenze der mobilen Wetterstation ersichtlich. Innerhalb der Systemgrenze befinden sich die Photovoltaikanlage, der Akkumulator und die passiven Elemente der mobilen Wetterstation. Ausserhalb der Systemgrenze befindet sich das Stromnetz. Durch die Pfeile wird der Energiefluss aufgezeigt, welcher von der Photovoltaikanlage und vom Stromnetz in den Akkumulator fliesst und von dort aus die passiven Elemente (MCU, Sensorik, RTC, Datenspeicherung, Kommunikationsmodule) der mobilen Wetterstation speist.\\
Der Akku bildet das Kernstück der Energieversorgung, da dieser die Quelle für die mobile Wetterstation ist. Um diesen vor Schäden zu schützen, ist eine Ladeschaltung notwendig, welche eine Überladungs- und Unterladungsschutz beinhaltet. Die Betriebszeit der mobilen Wetterstation hängt im wesentlichen von der Betriebszeit des Akkumulators ab. Um die Betriebszeit des Akkumulators zu verbessern, soll eine Photovoltaikanlage diesen während des Betriebs laden. Ausserdem soll der Akkumulator ebenso mit einem Stecker über das Schweizer Stromnetz ladbar sein. Falls es dennoch zur kompletten Entladung oder zu einem Defekt des Akkumulators kommt, soll dieser im Idealfall leicht austauschbar sein, was als Wunschziel definiert wurde.\\
\subsubsection{Printherstellung}
Es wird in diesem Projekt ein 4-Lagen PCB designed und bestückt.\\
