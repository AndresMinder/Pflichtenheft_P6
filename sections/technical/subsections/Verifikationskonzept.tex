\subsection{Verifikationskonzept}
\todo[inline]{Hier wird geschrieben, wie die einzelnen Teile des Projekts verifiziert werden, sowie am Schluss geplant ist, das gesamte Projekt zu verifizieren.}
\subsubsection{Validierung der Sensorik}
In einem vorhergehenden Projekt wurde die Sensorik , bis auf den Sensor für die Sonnenstundenzählung, implementiert, jedoch nicht validiert. Dies soll in diesem Projekt nachgeholt werden.\\
\paragraph{Validierung des BME280}
Mit dem BME280 wird die Lufttemperarut, die Luftfeuchtigkeit und der Luftdruck gemessen.
\todo[inline]{Muss ergänzt werden.}
\paragraph{Validierung des Anemometers}
Mit dem Anemometer wird die Windgeschwindigkeit gemessen und die Stärke nach Beaufort-Skala eingestuft.
\todo[inline]{Muss ergänzt werden.}
\paragraph{Validierung der Windfahne}
Die Windfahne ermittelt die Windrichtung.
\todo[inline]{Muss ergänzt werden.}
\paragraph{Validierung des Ombrometers}
Das Ombrometer misst die Niederschlagsmenge nach dem Kipplöffelprinzip.
\todo[inline]{Muss ergänzt werden.}
\paragraph{Validierung der Sonnenstundenzählung}
\todo[inline]{Muss ergänzt werden.}
\subsubsection{Validierung der Kommunikationsmodule}
\todo[inline]{Muss ergänzt werden.}
\subsubsection{Validierung des RTC}
\todo[inline]{Muss ergänzt werden oder rausgenommen.}
\subsubsection{Validierung der Datenspeicherung}
\todo[inline]{Muss ergänzt werden oder rausgenommen.}
\subsubsection{Validierung der Energieversorgung}
Damit die Energieversorgung als funktionsfähig eingestuft werden kann, muss das Verhalten der Ladeschaltung mit dem Überladungs- und Unterladungsschutz getestet werden.\\
In einem ersten Schritt soll die Ladefunktion mit Überladungsschutz bei der Speisung mittels Stromnetz getestet werden. Dazu wird ein Versuchsaufbau im Labor aufgebaut, wobei das Laden des Akkumulators beobachtet wird.\\
In einem zweiten Schritt soll der Unterladungsschutz in einem weiteren Laborversuch getestet werden. Dazu wird eine Last an den Akkumulator gelegt und die Entladung beobachtet.\\
In einem dritten Schritt soll die Ladefunktion mit Überladungsschutz bei der Speisung mittels Photovoltaikanlage getestet werden. Dies muss unter freiem Himmel geschehen damit das reale Verhalten der Photovoltaikanlage betrachtet werden kann. Auch hier wird wieder das Ladeverhalten des Akkumulators betrachtet, wobei dies erst ohne und danach mit Last erfolgt.\\