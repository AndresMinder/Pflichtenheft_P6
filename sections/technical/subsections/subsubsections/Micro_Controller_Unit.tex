\subsubsection{Micro Controller Unit (MCU)}
\begin{wrapfigure}{r}{0.5\textwidth}
  \vspace{-10pt}
  \begin{center}
    \missingfigure[figwidth=0.38\textwidth]{das Bild wird eh ersetzt oder sogar komplett gelöscht}
  \end{center}
  \vspace{-10pt}
  \caption{Arduino Mega \cite{Elektronik}}
  \vspace{-10pt}
  \label{fig:arduino_mega}
\end{wrapfigure}
Für die \textit{MCU} wurde im Projekt 5 ein Microcontroller mit bereits vorhandener Peripherie verwendet, welcher ähnlich wie der in Abbildung \ref{fig:arduino_mega} ersichtliche Arduino Mega aufgebaut ist. In diesem Projekt wird ein separates Printed Circuit Board (PCB) für die \textit{MCU} designed, welche die benötigte Peripherie aufweist. Auf diese Art kann Platz und Gewicht gespart werden, was der Mobilität der mobilen Wetterstation zugute kommt.\\